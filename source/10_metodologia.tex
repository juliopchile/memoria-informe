\section{Metodología}
Para alcanzar los objetivos planteados en el proyecto se propone el siguiente flujo de trabajo:

\begin{enumerate}
    \item Adquirir o desarrollar una base de datos orientada a la segmentación de instancias.
    \item Adquirir o desarrollar una base de datos para tareas de seguimiento.
    \item Llevar a cabo el entrenamiento de diversos modelos YOLO para segmentación de instancias, utilizando hiperparámetros de entrenamiento por defecto con algunas variaciones.
    \item Realizar validación de los modelos previamente entrenados para obtener un ``benchmark'' inicial que compare los diferentes modelos en distintas circunstancias de entrenamiento, esto con el fin de encaminar la siguiente etapa.
    \item Realizar la validación de los modelos entrenados con el fin de establecer un ``benchmark'' inicial que permita comparar el desempeño de los distintos modelos en diversas condiciones de entrenamiento y orientar la siguiente etapa.
    \item Ejecutar una búsqueda de hiperparámetros para identificar la configuración óptima para el entrenamiento definitivo del modelo. Dicha búsqueda se restringe a los casos más prometedores identificados en la etapa anterior, con el objetivo de reducir el espacio de búsqueda debido a limitaciones de tiempo.
    \item Proceder al entrenamiento final del modelo, empleando los hiperparámetros optimizados, y validar los resultados obtenidos.
    \item Validar los modelos entrenados en la tarea de seguimiento.
\end{enumerate}

Cada etapa conlleva desafíos específicos, requiere el uso de herramientas particulares y demanda el desarrollo correspondiente de código. Además, con el objetivo de facilitar la replicación de los experimentos y resultados obtenidos en este trabajo, se contempla de forma paralela el entrenamiento y validación utilizando la base de datos pública Deepfish; en dicha parte de la memoria no se modifica la base de datos ni se incluyen pruebas relacionadas con seguimiento.

\section{Herramientas computacionales}
En esta sección se discuten los programas y herramientas utilizadas en el desarrollado de esta memoria.

\subsection{Herramientas de desarrollo y despliegue}
Python se ha consolidado como el lenguaje de programación por excelencia para el desarrollo y despliegue de modelos de aprendizaje automático. Entre los ``frameworks'' principales destacan TensorFlow y PyTorch, las cuales se caracterizan por su robustez y adaptabilidad a diversas necesidades. Dado que esta memoria consiste en entrenar un modelo de segmentación y no modificar ni diseñar uno nuevo, es recomendable emplear una librería más sencilla para esta tarea.

\begin{table}[htbp]
\centering
\caption[Plataformas de desarrollo en Python para visión por computadora]{Tabla comparativa de distintas plataformas de desarrollo en Python.}
\label{tab:comparativa_plataformas_desarrollo}
\resizebox{\textwidth}{!}{
\begin{tabular}{|l|l|l|l|l|}
\hline
\multicolumn{1}{|c|}{\textbf{Característica}} & \multicolumn{1}{c|}{\textbf{Tensorflow} \cite{TensorFlow,TensorFlow-whitepaper}} & \multicolumn{1}{c|}{\textbf{Pytorch} \cite{Pytorch,Pytorch2}} & \multicolumn{1}{c|}{\textbf{Supergradients} \cite{Supergradients}} & \multicolumn{1}{c|}{\textbf{Ultralytics} \cite{Ultralytics}} \\ \hline
\begin{tabular}[c]{@{}l@{}}Disponibilidad\\ de modelos\end{tabular} & \begin{tabular}[c]{@{}l@{}}\checkmark\hspace{3px} Múltiples modelos\\ y backbones para usar.\\ \checkmark\hspace{3px} Permite customizar y\\ diseñar modelos.\end{tabular} & \begin{tabular}[c]{@{}l@{}}\checkmark\hspace{3px} Multiples modelos\\ y backbones para usar.\\ \checkmark\hspace{3px} El mayor nivel de\\ customización.\end{tabular} & \begin{tabular}[c]{@{}l@{}}\checkmark\hspace{3px} Modelos de pose y de\\ detección.\\ \xmark \hspace{3px} No hay soporte para\\ modelos de segmentación.\end{tabular} & \begin{tabular}[c]{@{}l@{}}\checkmark\hspace{3px} Modelos YOLO\\ para múltiples tareas.\\ \xmark \hspace{3px} Poca a nula\\ customización.\end{tabular} \\ \hline
Entrenamiento & \begin{tabular}[c]{@{}l@{}}\checkmark\hspace{3px} Simple de realizar.\\ \checkmark\hspace{3px} Muy alto nivel de\\ personalización.\end{tabular} & \begin{tabular}[c]{@{}l@{}}\checkmark\hspace{3px} Algo complicado\\ de realizar.\\ \checkmark\hspace{3px} El mayor nivel de\\ personalización.\end{tabular} & \begin{tabular}[c]{@{}l@{}}\checkmark\hspace{3px}Algo complicado\\ de realizar.\\ \checkmark\hspace{3px} Alto nivel de personaliza-\\ ción (en modelos soportados).\end{tabular} & \begin{tabular}[c]{@{}l@{}}\checkmark\hspace{3px} Muy sencillo.\\ \checkmark\hspace{3px} Buen nivel de\\ personalización.\end{tabular} \\ \hline
Validación & \begin{tabular}[c]{@{}l@{}}\checkmark\hspace{3px} Implementación\\ manual.\end{tabular} & \begin{tabular}[c]{@{}l@{}}\checkmark\hspace{3px} Implementación\\ manual.\end{tabular} & \begin{tabular}[c]{@{}l@{}}\checkmark\hspace{3px} Implementación\\ por interfaz.\end{tabular} & \begin{tabular}[c]{@{}l@{}}\checkmark\hspace{3px} Implementación\\ por interfaz.\end{tabular} \\ \hline
\begin{tabular}[c]{@{}l@{}}Exportación\\ con TensorRT\end{tabular} & \checkmark\hspace{3px} Implementado. & \checkmark\hspace{3px} Implementado. & \begin{tabular}[c]{@{}l@{}}\xmark \hspace{3px} Debe implementarse\\ a parte utilizando Pytorch.\end{tabular} & \checkmark\hspace{3px} Wrapp de Pytorch. \\ \hline
Dificultad. & Media & Alta & Media-alta & Fácil \\ \hline
\end{tabular}
}
\end{table}

\subsubsection{Ultralytics}
Ultralytics \cite{Ultralytics} es un ``toolkit'' construido sobre PyTorch, especializado en la implementación de la familia de modelos YOLO para detección, segmentación y estimación de pose en tiempo real. Proporciona ``scripts'' listos para usar, configuraciones predeterminadas optimizadas y facilidades para búsqueda de hiperparámetros, exportación a formatos ONNX y TensorRT, y despliegue en dispositivos borde. Su enfoque ligero y modular permite integrarlo fácilmente en ``pipelines'' de producción donde se requiera alta velocidad de inferencia y versatilidad en el entrenamiento. El mismo equipo de Ultralytics ha desarrollado algunos de los últimos modelos YOLO disponibles, y es esta disponibilidad de dichos modelos, al igual que la facilidad en sus entrenamientos, la razón por la que se decide usar esta biblioteca por sobre otras alternativas ---como SuperGradients.

\subsection{Herramientas de etiquetado}

\begin{table}[htpb]
\centering
\caption[Plataformas de etiquetado de bases de datos]{Tabla comparativa de distintas plataformas de etiquetado.}
\label{tab:comparativa_plataformas_etiquetado}
\resizebox{\textwidth}{!}{
\begin{tabular}{|l|l|l|l|l|l|}
\hline
\multicolumn{1}{|c|}{\textbf{Plataforma}} & \multicolumn{1}{c|}{\textbf{Bbox}} & \multicolumn{1}{c|}{\textbf{Segmentación}} & \multicolumn{1}{c|}{\textbf{Tracking}} & \multicolumn{1}{c|}{\textbf{Pros}} & \multicolumn{1}{c|}{\textbf{Contras}} \\ \hline
Roboflow \cite{Roboflow} & \begin{tabular}[c]{@{}l@{}}\checkmark\hspace{3px} Manual\\ \xmark \hspace{3px} Asistida\end{tabular} & \begin{tabular}[c]{@{}l@{}}\checkmark\hspace{3px} Manual\\ \checkmark\hspace{3px} Asistida\end{tabular} & \xmark \hspace{3px} No disponible. & \begin{tabular}[c]{@{}l@{}}- Interfaz simple.\\ - Permite entrenar \\ modelos dentro\\ de su plataforma.\end{tabular} & \begin{tabular}[c]{@{}l@{}}- Hay que pagar\\ para tener el\\ dataset privado.\end{tabular} \\ \hline
CVAT \cite{CVAT} & \begin{tabular}[c]{@{}l@{}}\checkmark\hspace{3px} Manual\\ \xmark \hspace{3px} Asistida\end{tabular} & \begin{tabular}[c]{@{}l@{}}\checkmark\hspace{3px} Manual\\ \checkmark\hspace{3px} Asistida\end{tabular} & \begin{tabular}[c]{@{}l@{}}\checkmark\hspace{3px} Manual\\ \checkmark\hspace{3px} Asistida\\ con bbox.\end{tabular} & \begin{tabular}[c]{@{}l@{}}- Intefaz simple.\\ -Corre de forma\\ local.\end{tabular} & \begin{tabular}[c]{@{}l@{}}- Puede tener\\ incompatibilidad\\ con Firefox.\\ - Segmentación\\ asistida lenta.\end{tabular} \\ \hline
V7 Labs \cite{V7Labs} & \begin{tabular}[c]{@{}l@{}}\checkmark\hspace{3px} Manual\\ \checkmark\hspace{3px} Asistida\end{tabular} & \begin{tabular}[c]{@{}l@{}}\checkmark\hspace{3px} Manual\\ \checkmark\hspace{3px} Asistida\end{tabular} & \begin{tabular}[c]{@{}l@{}}\checkmark\hspace{3px} Manual\\ \checkmark\hspace{3px} Asistida\\ con segmentación.\end{tabular} & \begin{tabular}[c]{@{}l@{}}- Permite entrenar\\ modelos dentro\\ de su plataforma.\end{tabular} & \begin{tabular}[c]{@{}l@{}}- Interfaz muy\\ lenta y con lag.\end{tabular} \\ \hline
Labellerr \cite{Labellerr} & \begin{tabular}[c]{@{}l@{}}\checkmark\hspace{3px} Manual\\ \checkmark\hspace{3px} Asistida\end{tabular} & \begin{tabular}[c]{@{}l@{}}\checkmark\hspace{3px} Manual\\ \checkmark\hspace{3px} Asistida\end{tabular} & \begin{tabular}[c]{@{}l@{}}\checkmark\hspace{3px} Manual\\ \xmark \hspace{3px} Asistida\end{tabular} & \begin{tabular}[c]{@{}l@{}}- Permite entrenar\\ modelos dentro de\\ su plataforma.\end{tabular} & \begin{tabular}[c]{@{}l@{}}- Interfaz\\ complicada y\\ difícil de usar.\end{tabular} \\ \hline
Supervisely \cite{Supervisely} & \begin{tabular}[c]{@{}l@{}}\checkmark\hspace{3px} Manual\\ \checkmark\hspace{3px} Asistida\end{tabular} & \begin{tabular}[c]{@{}l@{}}\checkmark\hspace{3px} Manual\\ \checkmark\hspace{3px} Asistida\end{tabular} & \begin{tabular}[c]{@{}l@{}}\checkmark\hspace{3px} Manual\\ \checkmark\hspace{3px} Asistida\end{tabular} & \begin{tabular}[c]{@{}l@{}}- Múltiples apps\\ y opciones para\\ configurar los\\ proyectos.\end{tabular} & \begin{tabular}[c]{@{}l@{}}- Tracking\\ asistido lento.\end{tabular} \\ \hline
SAM2 \cite{SAM2} & \checkmark\hspace{3px} Asistida & \checkmark\hspace{3px} Asistida & \checkmark\hspace{3px} Asistida & \begin{tabular}[c]{@{}l@{}}- Se ejecuta\\ localmente\end{tabular} & \begin{tabular}[c]{@{}l@{}}- Implementación\\ manual.\end{tabular} \\ \hline
\end{tabular}
}
\end{table}

\subsubsection{CVAT}
CVAT \cite{CVAT} (Computer Vision Annotation Tool) es una herramienta de código abierto diseñada para facilitar la anotación de imágenes y videos. Soporta la creación de cajas englobantes, polígonos, y keypoints de manera manual, así como la segmentación de instancias y seguimiento de objetos en videos. CVAT es conocido por su simplicidad y velocidad, lo que lo convierte en una opción ideal para proyectos que requieren anotaciones precisas. Sin embargo, la segmentación automática puede ser lenta en datasets grandes, y pueden existir incompatibilidades con el navegador Firefox.