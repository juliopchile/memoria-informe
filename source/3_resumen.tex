\pagestyle{fancy}
\fancyhead[L]{\includegraphics[scale=0.2]{logos/elo_logo.pdf}}
\fancyhead[R]{\includegraphics[scale=0.12]{logos/elo_iso.png}}

\begin{center}
    {\singlespacing\Large{\textbf{Segmentación de instancias y seguimiento basado en detección para caracterización de biomasa y salud en la salmonicultura}}}\\

    \normalsize
    \textbf{Julio Eduardo López Blanche}\\

    Memoria para optar al título de Ingeniero Civil Electrónico, Mención en Computadores y Sub-Mención en Telecomunicaciones.\\

    \textbf{Universidad Técnica Federico Santa María}\\

    Profesor Guía: Dr. Marcos Zúñiga Barraza\\

    \monthyeardate\today
\end{center}

\begin{poliabstract}{Resumen} 
\normalsize
En la industria piscícola, el monitoreo constante de la salud de los peces es crucial. Gracias a los avances en visión por computadora es posible realizar esta labor de forma escalable y menos invasiva. WildSense, empresa spin-off de la UTFSM, provee servicios para la estimación de masa en salmones, donde la segmentación de instancias y seguimiento basado en detección son parte fundamental de su ``pipeline'', pero aún presenta oportunidades de optimización.

Este proyecto perfecciona una base de datos de segmentación de instancias de salmones, al depurarla para incluir únicamente casos de interés, lo que permite entrenar modelos YOLO con rendimientos superiores a trabajos previos. Se optimizan los hiperparámetros durante el entrenamiento y se exportan los modelos a TensorRT para reducir los tiempos de inferencia.

Los resultados demuestran que una base de datos más precisa mejora la calidad de los modelos, la optimización de hiperparámetros produce mejores resultados y la conversión a TensorRT reduce significativamente los tiempos de inferencia, con mínima pérdida de desempeño.
\end{poliabstract}

\normalsize
\keywordses{Visión por computadora, segmentación de instancias, seguimiento basado en detección, aprendizaje de máquina, aprendizaje profundo, modelos convolucionales, hiperparámetro, YOLO, cuantización numérica, TensorRT.}

\newpage

\selectlanguage{english}
\begin{center}
    {\singlespacing\Large{\textbf{Instance segmentation and tracking by detection for biomase and health caracterization in aquaculture of salmonids.}}}\\

    \normalsize
    \textbf{Julio Eduardo López Blanche}\\

    Final Project Report for the Electronic Civil Engineering degree with Minor in Computers and Secondary Minor in Telecommunications.\\

    \textbf{Universidad Técnica Federico Santa María}\\

    Advisor: Dr. Marcos Zúñiga Barraza\\

    \monthyeardate\today
\end{center}

\begin{poliabstract}{Abstract} 
\normalsize
In the aquaculture industry, the constant monitoring of fish health is crucial. Advances in computer vision now allow this task to be performed in a scalable and less invasive manner. WildSense, a spin-off company of the UTFSM, provides services for weight estimation in salmon, for which instance segmentation and detection-based tracking are integral components of its pipeline, although there remains room for optimization.

This project refines an instance segmentation database for salmon by filtering it to include only relevant cases, thereby enabling the training of YOLO models with performance superior to previous works. Hyperparameters are optimized during training, and the models are exported to TensorRT in order to reduce inference times.

The results demonstrate that a more precise database enhances the quality of the models, hyperparameter optimization yields better outcomes, and the conversion to TensorRT significantly reduces inference times with minimal performance loss.
\end{poliabstract}

\normalsize
\keywordsen{Computer vision, instance segmentation, tracking by detection, machine learning, deep learning, convolutional models, hyperparameter, YOLO, numerical quantization, TensorRT.}

\selectlanguage{spanish}